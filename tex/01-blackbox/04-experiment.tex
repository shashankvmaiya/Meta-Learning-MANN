\item \points{4} {\bf Experimentation}

\begin{enumerate}[label=\alph*]
    \item Experiment with one hyperparameter that affects the performance of the model, such as the type of recurrent layer, size of hidden state, learning rate, or number of layers. 
    
    The plot that shows how the meta-test query set classification accuracy of the model changes on 1-shot, 3-way classification as you change the parameter should look as follows.

    \begin{center}
        \includegraphics[width = 0.75\textwidth]{./figures/soln4i}
    \end{center}
    
    Provide a brief rationale for why you chose the parameter and what you observed in the caption for the plot.    
    
    \item \textbf{Extra Credit:} In this question we'll explore the effect of memory representation on model performance. We will focus on the $K=1$, $N=3$ case.
    
    In the previous experiments we used an LSTM model with 128 units. Consider additional memory sizes of 256, and 8. 
    
    The plots should look as follows:
    \begin{center}
        \includegraphics[width = 0.75\textwidth]{./figures/soln4ii}
    \end{center}
    
    How does increasing and decreasing the memory capacity influence performance?
        
\end{enumerate}